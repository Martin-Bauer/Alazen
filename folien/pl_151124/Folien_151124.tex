%Danke an Samuel Brack für die Vorlage
\documentclass{beamer}
\usepackage{ngerman}
\usepackage{hyperref}
\usepackage[utf8]{inputenc}
\usepackage{amsmath,amsfonts,amssymb,dsfont}
\usepackage[ngerman]{babel}
\usepackage[autostyle=true,german=quotes]{csquotes}
\usepackage{graphicx}
\usepackage{multirow}
\usepackage{listings}
\usepackage{hyperref}
\usepackage{tikz}
\usetikzlibrary{automata}
\usepackage{svg}
\usetheme{HUmetrics}
\usecolortheme{default}
\usefonttheme{serif}
\setbeamertemplate{navigation symbols}{}
\lstset{language=[LaTeX]TeX,
	basicstyle=\tiny,
	stringstyle=\ttfamily,
	commentstyle=,
	tabsize=2,
	numbers=left,
    numberstyle=\tiny,
    numbersep=5pt,
	frame=single,
	framerule=0.1pt,
	keywordstyle=\color{blue},
    xleftmargin=15pt,
	xrightmargin=15pt,
    literate={ä}{{\"a}}1 {Ü}{{\"U}}1 {~}{{\textasciitilde}}1
}


\title[VGB]{Semesterprojekt Verteiltes Genom Browsing}
\subtitle{Git}
\author[Kruse,Schumacher]{Malte Kruse\\Ben Schumacher}
\institute{Institut für Informatik\\Humboldt-Universität zu Berlin}
\date{24.11.2015}

\begin{document}



% Titelseite
\begin{frame}
	\titlepage
\end{frame}


% Welche Repositorys verwenden wir?
\begin{frame}
	\frametitle{Welche Repositorys verwenden wir?}
	Je eins für den Quellcode jedes Teilprojekts
	\begin{itemize}
		\item \url{github.com/hu-semesterprojekt-genombrowser/Alazen-Integration}
		\item \url{github.com/hu-semesterprojekt-genombrowser/Alazen-Middleware}
		\item \url{github.com/hu-semesterprojekt-genombrowser/Alazen-GUI}
	\end{itemize}
\end{frame}

% Welche Repositorys verwenden wir?
\begin{frame}
	\frametitle{Welche Repositorys verwenden wir?}
	Ein gesammt Repository\\
	\url{github.com/hu-semesterprojekt-genombrowser/Alazen}\\
	für
	\begin{itemize}[<+->]
		\item Software Entwürfe
		\item Wiki
		\item Folien
		\item Lauffähige Versionen
	\end{itemize}
\end{frame}

% Branches
\begin{frame}
	\frametitle{Branches}
	Warum?
	\begin{itemize}
		\item Zu viele Menschen sollen nicht gleichzeitig an einer Datei arbeiten
		\item Neue Features sollen von allen kommentiert werde können (Merge Requests)
	\end{itemize}
\end{frame}

% Branches
\begin{frame}
	\frametitle{Konzept}
	\begin{figure}
	    \centering
	    \def\svgwidth{0.8\columnwidth}
	    \input{image.pdf_tex}
	\end{figure}
\end{frame}


% Branches
\begin{frame}
	\frametitle{Nützliche Links}
	Eine sehr gute Giteinführung
	\begin{itemize}
		\item \url{https://www.atlassian.com/git/tutorials/}
	\end{itemize}
	Git Dokumentation
	\begin{itemize}
		\item \url{https://git-scm.com/doc}
	\end{itemize}
\end{frame}

\end{document}