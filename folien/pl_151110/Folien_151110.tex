%Danke an Samuel Brack für die Vorlage
\documentclass{beamer}
\usepackage{ngerman}
\usepackage[utf8]{inputenc}
\usepackage{amsmath,amsfonts,amssymb,dsfont}
\usepackage{graphicx}
\usepackage{multirow}
\usepackage{listings}
\usepackage{hyperref}
\usepackage{tikz}
\usetikzlibrary{automata}
\usetheme{HUmetrics}
\usecolortheme{default}
\usefonttheme{serif}
%\useinnertheme{circles}
\setbeamertemplate{navigation symbols}{}
\lstset{language=[LaTeX]TeX,
	basicstyle=\tiny,
	stringstyle=\ttfamily,
	commentstyle=,
	tabsize=2,
	numbers=left,
    numberstyle=\tiny,
    numbersep=5pt,
	frame=single,
	framerule=0.1pt,
	keywordstyle=\color{blue},
    xleftmargin=15pt,
	xrightmargin=15pt,
    literate={ä}{{\"a}}1 {Ü}{{\"U}}1 {~}{{\textasciitilde}}1
}


\title[VGB]{Semesterprojekt Verteiltes Genom Browsing}
\subtitle{Projektplan\\Wahl des Projektnames}
\author[Kruse,Schumacher]{Malte Kruse\\Ben Schumacher}
\institute{Institut für Informatik\\Humboldt-Universität zu Berlin}
\date{10.11.2015}

\begin{document}

% Titelseite
\begin{frame}
	\titlepage
\end{frame}

% Projektspezifikation
\begin{frame}
	\frametitle{Projektspezifikation}
	30.11.2015:
	\begin{itemize}
	  \item Modelle aller Teilprojekte (Schemata, UML, etc.) 
	  \item Exakte Beschreibung Integrationsprozess (welche Quelle, Attributauswahl und -mapping,  Mengengerüst, Modellierung des DWH)
	  \item Schnittstellenspezifikation (DB-Middelware, Middleware-Frontend)
	  \item Definition Komponententests (Unit-Tests), Stresstests (Middleware), Integrationstests
	  \item Formale Beschreibung Indexstruktur, Operationen als Pseudocode, Komplexitätsabschätzungen
	  \item Mock-Ups Benutzerschnittstelle
	  \item Sequence-charts für den Ablauf jeder zentralen Funktionen vom UI zur Infrastruktur und zurück
	\end{itemize}
\end{frame}

% 2. Meilenstein
\begin{frame}
	\frametitle{\underline{2. Meilenstein}: Lauffähige Version}
	7.12.2015:
\begin{itemize}
 \item Erste, mit (Test-)daten, lauffähige Version des Programms\\
 (GUI, Middleware, Integration)
  \begin{itemize} 
   \item Erste Version des Index, muss nicht parallel laufen
   \item Mit Testdaten gefülltes DWH
   \item Suchanfragen, Darstellung in Lanes, Referenzgenomanzeige
  \end{itemize}
 \item Zwischenstand Dokumentation an Projektleitung
\end{itemize}
\end{frame}

\begin{frame}
 \frametitle{}
 12.12.2015:
 \begin{itemize}
  \item Lauffähige Version des Programms mit mind. einer eingebundenen Hauptdatenbank
  \item Erste Version mit einem verbesserten Ansatz des verteilten Index.
  \item Der Index läuft parallel 
  \item Verbessertes GUI, bei der fehlerhafte Benutzereingaben vollständig behandelt werden.
  \item Bericht der Teilgruppenleiter über den Stand ihrer Teilprojekte
  \item Zwischenreview: Stand aller Teilprojekte 
 \end{itemize}
\end{frame}

% \underline{3. Meilenstein}
\begin{frame}
 \frametitle{\underline{3. Meilenstein}}
 19.12.2015:
 \begin{itemize}
 \item Lauffähige Version des Programms mit mind. zwei eingebundenen Hauptdatenbanken
 \item Eine erste Version der Datenbankintegrationsmöglichkeit für Benutzer steht zur Verfügung
 \item Der verteilte Index läuft auf allen 4 VMs
 \item Die GUI bietet erste Konfigurationsmöglichkeiten 
 \end{itemize}
\end{frame}

% Name
\begin{frame}
 \frametitle{Projektname}
 \begin{itemize}[<+->]
 \item The Mutation Viewer/MutaView
 \item The Mutation Lens
 \item The Mutation Browser/MutaBrowse/MutaSearch
 \item Alazen (Abgeleitet von Alhazen)
 \item AlazenViewer/Browser/Search
 \item GMB/GenMutationBase
 \end{itemize}
\end{frame}

\end{document}

