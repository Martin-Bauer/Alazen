\section{Schnittstellenspezifikation: Middleware - Frontend}
Bei der Kommunikation zwischen Middleware und Forntend ist ein Austausch von JSON über die HTTP-Methode GET vorgesehen.
An die Middleware werden folgende Informationen geschickt:
\\
\textit{Quellen; Chromosom; Position; Suche; Präfix-Flag; Zoomstufe; Anzahl der Subintervalle; HasDetail-Flag}.\\
Von der Middleware werden bei bestimmten Flags, folgende Daten erwartet:
\\
Wenn \textit{IsPräfix-Flag} = \texttt{true}: Array aller Präfixe\\
Wenn \textit{IsPräfix-Flag} = \texttt{false}: Name und Position der Suche\\[1em]
Wenn \textit{HasDetail-Flag} = \texttt{true}: RefSeq + Mutationen mit MetaDaten\\
Wenn \textit{HasDetail-Flag} = \texttt{false}: aggregierte Mutationen (durch Zoomstufe+Anzahl der Subintervalle)

\subsubsection{Suchanfrage JSON}
GUI $\Rightarrow$ Middleware
\begin{lstlisting}[language=c,
commentstyle=\fontsize{12}{14.4}\ttfamily,
basicstyle=\ttfamily\fontsize{10}{12}\selectfont, showstringspaces=false]
{
  "source": ["src1", "src2", ...],
  "chromosome": int x,
  "search": String a
  "isPrefix": (true|false)
}
\end{lstlisting}
GUI $\Leftarrow$ Middleware
\begin{lstlisting}[language=c,
commentstyle=\fontsize{12}{14.4}\ttfamily,
basicstyle=\ttfamily\fontsize{10}{12}\selectfont, showstringspaces=false]
{
  "source": ["src1", "src2", ...],
  "chromosome": int x,
  "search": String a,
  "prefix": ["String a", "String b", ...],
  "position": {"from": int x, "to": int y}
}
\end{lstlisting}
\paragraph{Bemerkung} Wenn das \textit{isPrefix} den Wert \texttt{'true'} besitzt, wird \texttt{"prefix"} befüllt mit einem Array aller möglichen Präfixe.\\
Wenn das \textit{isPrefix} den Wert \texttt{'false'} besitzt, wird \texttt{"prefix"} nicht befüllt, aber \texttt{"'search"} und \texttt{"position"} werden dann befüllt.
\newpage

\subsubsection{Intervallanfrage JSON}
GUI $\Rightarrow$ Middleware
\begin{lstlisting}[language=c,
commentstyle=\fontsize{12}{14.4}\selectfont,
basicstyle=\ttfamily\fontsize{10}{12}\selectfont, showstringspaces=false]
{
  "source": String s,
  "chromosome": int x,
  "position": {"from": int x, "to": int y},
  "zoom": int y,
  "subindex": int z,
  "hasDetail": (true|false)
}
\end{lstlisting}
GUI $\Leftarrow$ Middleware (pro Quelle)
\begin{lstlisting}[language=c,
commentstyle=\fontsize{12}{14.4}\selectfont,
basicstyle=\ttfamily\fontsize{10}{12}\selectfont, showstringspaces=false]
{
  {
    "source:" String s,
    "chromosome": int x,
    "position": {"from": int x, "to": int y},
    "details": { "refseq": String b,
                "mutations": [{ "name": String s,
                                "position": {"from": int x, "to": int y},
                                "metadata": {...}
                              },{...},...]},
    "graph": {
              { "subintervall": int x,
                "counts": int y
              },
              {...}
            }
  }
}
\end{lstlisting}
\paragraph{Bemerkung} Wenn das \textit{HasDetail-Flag} \texttt{'true'} ist, dann wird \texttt{"detail"} befüllt und \texttt{"graph"} bleibt leer.\\
Wenn das \textit{HasDetail-Flag} \texttt{'false'} ist, dann wird \texttt{"graph"} befüllt und \texttt{"detail"} bleibt leer.\\
Mit \texttt{"'subindex"} ist die Anzahl der Subintervalle gemeint.
\newpage
\subsection{Beispiel der Kommunikation mit JSON}
\paragraph{Suchanfrage} Es wird eine Suchanfrage des Users getätigt. Die \textit{Suche} wird mit dem Wert \textit{"FOXP2"} befüllt.\\
\newline
GUI $\Rightarrow$ Middleware
\begin{lstlisting}[language=c,
commentstyle=\fontsize{12}{14.4}\selectfont,
basicstyle=\ttfamily\fontsize{10}{12}\selectfont, showstringspaces=false]
{
  "search": "FOXP2"
  "isPrefix": false
}
\end{lstlisting}
GUI $\Leftarrow$ Middleware
\begin{lstlisting}[language=c,
commentstyle=\fontsize{12}{14.4}\selectfont,
basicstyle=\ttfamily\fontsize{10}{12}\selectfont, showstringspaces=false]
{
  "chromosome": 7,
  "search": "FOXP2",
  "prefix": [],
  "position": {"from": 114086310, "to": 114693772}
}
\end{lstlisting}
Darauf folgt eine Intervallanfrage mit der \textit{Zoomstufe} von \texttt{'1,000,000bp'} und der Anzahl der Subintervalle von \texttt{'200'}.\\
\newline
GUI $\Rightarrow$ Middleware
\begin{lstlisting}[language=c,
commentstyle=\fontsize{12}{14.4}\selectfont,
basicstyle=\ttfamily\fontsize{10}{12}\selectfont, showstringspaces=false]
{
  "source": "Source 1",
  "chromosome": 7,
  "position": {"from": 114086310, "to": 114693772},
  "zoom": 1,000,000,
  "subindex": 200,
  "hasDetail": false
}
\end{lstlisting}
GUI $\Leftarrow$ Middleware
\begin{lstlisting}[language=c,
commentstyle=\fontsize{12}{14.4}\selectfont,
basicstyle=\ttfamily\fontsize{10}{12}\selectfont, showstringspaces=false]
{
  {
    "source:" "Source 1",
    "chromosome": 7,
    "position": {"from": 114086310, "to": 114693772},
    "details": {},
    "graph": {
              { "subintervall": 114091310,
                "counts": 233
              },
              {
                "subintervall": 114096310,
                "counts": 122
              },
              {
                "subintervall": 114101310,
                "counts": 89
              },
              {
                "subintervall": 114106310,
                "counts": 234
              },
              {...},
              ...
            }
  }
}
\end{lstlisting}

\paragraph{Intervallsuche} Der User gibt ein \textit{Intervall} an und möchte sich in der niedrigsten \textit{Zoomstufe} die Details anschauen.\\
\newline
GUI $\Rightarrow$ Middleware
\begin{lstlisting}[language=c,
commentstyle=\fontsize{12}{14.4}\selectfont,
basicstyle=\ttfamily\fontsize{10}{12}\selectfont]
{
  "source": "Source 1",
  "chromosome": 14,
  "position": {"from": 5465, "to": 5665},
  "zoom": 200,
  "subindex": 0,
  "hasDetail": true
}
\end{lstlisting}
GUI $\Leftarrow$ Middleware
\begin{lstlisting}[language=c,
commentstyle=\fontsize{12}{14.4}\selectfont,
basicstyle=\ttfamily\fontsize{10}{12}\selectfont,showstringspaces=false]
{
  {
    "source:" Source 1,
    "chromosome": 14,
    "position": {"from": 5465, "to": 5665},
    "details": { "refseq": "CCCTGAGAAAGACGCCCTGAGCGGGTAACGTCAACTGTCG
			    TGGGATTGGAAGAACTTGGTGGGGTTCCACATATCATGAT
                            CACGTAGAACACAATAAAGAAAATCTCCGGAGCGTGAATT
                            AAACTGAACTGACTCTTGTCAGTGTCTCGCTTACCACGTC
                            AAGGAAGTCAATGATGCCCTAATCTGGATAGCTTTGTCAT",
                "mutations": 
                [{ "name": "RS12334",
                                "position": {"from": 5480, "to": 5498},
                                "metadata": {"sequence": "TCGGTATTCTACGTCTGC"}
                              },
                 { "name": "RBDDFG23",
                                "position": {"from": 5560, "to": 5567},
                                "metadata": {"sequence": "GGGCACT"}
                              }
                 { "name": "RS12334",
                                "position": {"from": 5599, "to": 5610},
                                "metadata": {"sequence": "TGACGAGATCG"}
                              }
                 { "name": "RS12334",
                                "position": {"from": 5650, "to": 5655},
                                "metadata": {"sequence": "CGACC"}
                              }]
              },
    "graph": {}
  }
}
\end{lstlisting}
