\documentclass{scrartcl}

\usepackage[utf8]{inputenc}
\usepackage[T1]{fontenc}
\usepackage{lmodern}
\usepackage[ngerman]{babel}
\usepackage{amsmath, amsfonts, amssymb}
\usepackage{graphicx}
\usepackage [german=quotes] {csquotes}
\usepackage {soul}
\usepackage{fancyhdr} %Header
\usepackage{booktabs} %dicke Linien für Tabellen
\usepackage{longtable}
\usepackage{listings}

\renewcommand{\headrulewidth}{0pt}

\title{Verteiltes Genom Browsing}
\subtitle{Java-Styleguide}
\author{Malte Kruse, Ben Schumacher}

\setkomafont{subtitle}{\bfseries \LARGE}


\begin{document}
\pagestyle{fancy}
%\fancyhf{}
\cfoot{}
\rfoot{\pagemark}
%\lhead{Ben Schumacher (560493)\\Übungsblatt \arabic{nummerDesBlattes}\\Mittwoch 11-13 Uhr }
%\rhead{\textsc{Humboldt-Universität zu Berlin,\\
%Institut für Informatik}\\
%\textbf{Logik in der Informatik}}
%\hfill
%\section*{Projektplan}
\maketitle
\thispagestyle{empty}
\newpage

\begin{itemize}
\item Einrückung nur mit Leerzeichen, keine Tabs. Je zwei Leerzeichen pro Einrückung.
\item  Klassennamen groß und als CamelCase, z.B.:
\begin{lstlisting}
MyNewClassWithAVeryLongAndMeaninglessName
\end{lstlisting}
\item  Namen von Funktionen und Methoden Klein und CamelCase, z.B.
\begin{lstlisting}
     myPublicMethod()
\end{lstlisting}
\item Variablennamen klein und CamelCase. Der erster Buchstabe beschreibt den\\
Datentype (x bei Klassen) z.B.
\begin{lstlisting}
	boolean b_IsValid
	MyClass x_MyVariable
\end{lstlisting}
\item Aussagekräftige Variablennamen
\item Variablen- und Methodennamen werden in Englisch gewählt
\item Die Dokumentation erfolgt auf Deutsch
\item Methoden in der Form
\begin{lstlisting}
	void myMethod() {
	  //do stuff
	  return;
	}
\end{lstlisting}
\item if bzw. else Bedingungen in der Form
\begin{lstlisting}
	if (true) {
	  //do stuff
	} else {
	  //do other stuff
	} 
\end{lstlisting}
\item static final Variablen in Uppercase, z.b.
\begin{lstlisting}
	MY_FINAL_NUMBER
\end{lstlisting}
\end{itemize}

\end{document}