\section{Frontend}
\subsection{Mock-Ups der Benutzerschnittstelle}
\includepicture[width=\textwidth]{gui/gb_mockup_detail_view.png}{Die Detail-Ansicht des Genom Browsers. In dieser Ansicht werden die Mutationen auf dem Referenz-Genom, fuer eine bessere Darstellung, abgebildet. Pro aktiver Quelle wird eine Lane angezeigt.}
\includepicture[width=\textwidth]{gui/gb_mockup_index_view.png}{Die Aggregierte-Ansicht des Genom Browsers. In dieser Ansicht werden die Mutationen aggregiert dargestellt, fuer eine bessere Uebersicht auf dem Genom.Pro aktiver Quelle wird eine Lane angezeigt.}
\subsection{Klassen-Diagramm}
\includepicture[width=\textwidth]{gui/gui-klassendiagrammv2.png}{Das Klassen-Diagramm der GUI.}
\paragraph{Erläuterung der Abbildung \ref{fig:Das Klassen-Diagramm der GUI.}}
Die \textit{GUIComponent}-Klasse beinhaltet alle Komponenten der GUI und stellt die Kommunikation unter den Komponenten bereit. Die \textit{DaraProvider}-Klasse stellt die Methoden, für die Kommunikation zu der Middleware, bereit. Durch die \textit{LaneContainer}-Klasse werden die \textit{Lanes} erstellt. 
\subsection{Sequenzdiagramm}
\includepicture{gui/GUI_Sequenzdiagrammv2.png}{Anhand des Sequenzdiagramm wird der Methodenaufruf verdeutlicht.}
\subsection{Use Cases}
\includepicture[width=\textwidth]{gui/gui-usecasediagrammv2.png}{Das Use-Case-Diagramm listet die Möglichkeiten der Interaktionen mit der GUI auf.}
\paragraph{Erläuterung des Use-Case (mit Bezug auf das Sequenzdiagramm)}
Der Benutzer hat die Möglichkeit mit der GUI zu interagieren. Durch das Suchfeld hat der Benutzer die Möglichkeit nach einem Gen zu \textbf{suchen} oder sich eine(n) \textbf{Position / Bereich anzeigen lassen}. Zur genaueren Auswahl kann dieser sich eine \textbf{Chromosom auswählen}. Durch das \textbf{hinzufügen / entfernen} einer oder mehrerer Quellen, kann der Benutzer das Angezeigte weiter eingrenzen. Durch das \textbf{verändern des Zoom-Bereich} kann der Benutzer die Anzahl der angezeigten Basenpaare bestimmen. Durch einfaches \textbf{verschieben der Lane} ist es dem Benutzer möglich, den genauen Bereich zu bestimmen, den dieser betrachten möchte.
\subsection{Unit-Tests}
\subsubsection{Suchfunktion}
\begin{enumerate}
	\item Wenn ich als Nutzer eine leere Suche starte, dann möchte ich eine entsprechende Fehlermeldung angezeigt bekommen.
	\item Wenn ich eine Suche mit falscher Eingabe starte, dann möchte ich eine entsprechende Fehlermeldung angezeigt bekommen.
	\item Wenn ich als Nutzer nach einem gültigen Intervall suche, dann wird automatisch die Zoomstufe auf dieses Intervall angepasst.
	\item Wenn ich als Nutzer nach einem vorhandenem Gene suche, dann wird automatisch die Zoomstufe auf dieses Intervall angepasst.
\end{enumerate}


\subsubsection{Quellen-Button}
\begin{enumerate}
	\item Wenn ich als Nutzer auf einen "Quellen"-Button drücke, dann wird mir die entsprechende Quelle zusätzlich zu den bereits dargestellten Quellen, angezeigt.
	\item Wenn ich als Nutzer auf den "Quellen"-Button einer bereits angezeigten Quelle drücke, wird die entsprechende Quelle nicht mehr angezeigt.
\end{enumerate}

\subsubsection{Quellen-Scroller}
\begin{enumerate}
	\item Als Nutzer kann ich mich über horizontales Scrolling synchron durch die Quellen bewegen.
\end{enumerate}

\subsubsection{Zoom-slider}
\begin{enumerate}
	\item Wenn ich die Zoomstufe über den Slider ändere, dann werden die Quellen entsprechend der eingestellten Stufe dargestellt.
	\item Wenn ich als Nutzer die feinste Zoomstufe einstelle, dann werden mir die Basenpaare angezeigt.
	\item Wenn ich als Nutzer eine andere Zoomstufe einstelle, dann werden mir aggregierten Daten angezeigt.
\end{enumerate}

\subsubsection{Chromosom-Auswahl}
\begin{enumerate}
	\item Als Nutzer kann ich über ein Dropdown aus einer Vorauswahl von Chromosomen auswählen.
	\item Wenn ich als Nutzer ein Chromosom auswähle, dann wird die Quellen-Anzeige automatisch entsprechend des ausgewählten Chromosoms aktualisiert.
	\item Wenn ich als Nutzer das bereits ausgewählte Chromosom erneut auswählen, dann passiert nichts.
\end{enumerate}

\subsubsection{Allgemein}
\begin{enumerate}
	\item Wenn ich als Nutzer auf eine Anfrage warten muss, wird mir dies durch einen Loading-Spinner signalisiert.	
\end{enumerate}
\newpage

%\section{Integrationstest}
%\subsection{Ablauf der Integrationstests}
