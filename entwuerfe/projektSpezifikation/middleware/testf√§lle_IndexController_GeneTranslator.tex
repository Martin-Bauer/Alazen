\documentclass[]{article}
\usepackage[utf8]{inputenc}



%opening
\title{Testfaelle GeneTranslator, IndexController}
\author{}

\begin{document}

\maketitle

\section{IndexController}
Um den IndexController zu testen wird ein Testprogramm genutzt, dass Anfragen stellt und das Ergebnis mit vorher manuell abgefragten Ergebnissen abgleicht:
\begin{verbatim}
import static org.junit.Assert.*;
import java.io.*;
import java.util.*;

public class IndexTest {
	public static void main(String[] args) throws IOException  {
		QueryHandler x_QH = new QueryHandler();
		File x_File = new File("./testQueries.txt");
		FileReader x_FR = new FileReader(x_File);
      	BufferedReader x_BR = new BufferedReader(x_FR);
      	String query[] = new String[1];
      	String answer = "";
      	String result = "";
      	Set<IntervalST.Node> x_Result = new HashSet<IntervalST.Node>();
      	while ((query[0] = x_BR.readLine()) != null) {
      		result = "";
      		if ((answer = x_BR.readLine()) != null) {
      			x_Result = x_QH.answerQuery(query);
      			for (IntervalST.Node x_Node : x_Result) {
      				result = result + x_Node.i_MutationID;
    			}
    			assertEquals(answer, result);		//compares expected answer and the result of the query
      		} else {
      			break;
      		}
      	}
	}
}
\end{verbatim}
Die Datei mit den Testdaten sieht folgendermaßen aus:
\begin{verbatim}
7775,7779
0
6463,6465
13
24141,24146
5
5323,5327
11
32445,32448
21
4535,4535
2
22550,22600
1
9700,10000
19
44000,45000
15
50000,60000
7
\end{verbatim}
\newpage
\end{document}

\section{GeneTranslator}
Der GeneTranslator hat 2 Aufgaben. Zum einen soll er während der Indexerstellung mit Inhalt (also Gennamen und zugeörigen Intervallen) befüllt werden, zum anderen soll eine Suche nach Gennamen in ihm möglich sein.

\textbf{Testfall 1 - addGene - Einfügen in Datenstruktur}\\
Eingabe: Testgen, 350, 500\\
Ausgabe: Dass Gen sollte in den Baum eingefügt sein und per searchForGene() findbar sein\\
\\
\textbf{Testfall 2 - addGene - Doppeltes Einfügen in Datenstruktur}\\
Eingabe: Testgen, 350, 500\\
		 Testgen, 350, 500\\
Ausgabe: Dass Gen sollte nur einmal in die Datenstruktur eingefügt werden. Ausgabe, dass das Gen bereits in der Struktur vorhanden ist.\\
\\
\textbf{Testfall 3 - addGene - Aufruf ohne Parameter}\\
Eingabe: [...]\\
Ausgabe: Fehlermeldung über  parameterlosen Aufruf\\
\\
\textbf{Testfall 4 - tranlateToIntervall - erfolgreiche Suche}\\
Eingabe: Testgen (befindet sich bereits in Datenstruktur)\\
Ausgabe: 350, 500\\
\\
\textbf{Testfall 5 - tranlateToIntervall - erfolglose Suche}\\
Eingabe: Testgen2 (befindet sich nich in Datenstruktur)\\
Ausgabe: Fehlermeldung über erfolglose Suche\\
\\
\textbf{Testfall 6 - tranlateToIntervall - Aufruf ohne Parameter}\\
Eingabe: [...]\\
Ausgabe: Fehlermeldung über Parameterlosen Aufruf\\
\\
\textbf{Testfall 7 - completeGeneName - erfolgreiche Suche mit einem Ergebnis}\\
Eingabe: Test (Testgen1 befindet sich bereits in Datenstruktur)\\
Ausgabe: Testgen1\\
\\
\textbf{Testfall 8 - completeGeneName - erfolgreiche Suche mit mehreren Ergebnissen}\\
Eingabe: Test (Testgen1 und Testgen2 befinden sich bereits in Datenstruktur)\\
Ausgabe: Testgen1, Testgen2\\
\\
\textbf{Testfall 9 - completeGeneName - erfolglose Suche}\\
Eingabe: Test (Es befindet sich kein Gen mit Präfix Test in der Datenstruktur)\\
Ausgabe: Fehlermeldung über erfolglose Suche
\end{document}

